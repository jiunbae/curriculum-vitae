\ifen{\section{Experience}}
\ifko{\section{경력}}

\outerlist{
    \entrybig
    {\link{https://rtzr.ai}{\textbf{Return Zero}}}{}
    {\ifen{Research Engineer}\ifko{연구 엔지니어}}{Jul. 2024 {\textendash} Present}
    \innerlist{
        \entry{\ifen{Researched and developed state-of-the-art speech-to-text engines based on NVIDIA Parakeet (FastConformer) for Korean and Japanese, achieving industry-leading performance across multiple metrics including WER and CER. Introduced LLM-based readability evaluation with a novel assessment dataset (100M+ sentences, validated on 10K+ human-annotated test samples) to quantitatively measure transcription quality that traditional metrics cannot capture.}\ifko{NVIDIA Parakeet (FastConformer) 기반으로 WER/CER 등 여러 지표에서 한국어/일본어 최고 수준의 STT 엔진을 연구 및 개발했으며, 새로운 LLM 기반 가독성 평가 데이터셋(1억+ 문장, 만+ 사람이 평가한 테스트 샘플로 검증)을 도입하여 기존 메트릭으로 측정할 수 없는 부분을 정량평가했습니다.}}
        \entry{\ifen{Led end-to-end development of LLM-powered features for Callabo (AI meeting transcription service), including translation, transcription correction, and real-time transcription, from research to production deployment.}\ifko{Callabo (AI 음성 회의록) 서비스의 LLM 기반 신규 기능(번역, 전사 교정, 실시간 전사)을 연구부터 배포까지 전담하여 개발했습니다.}}
        \entry{\ifen{Owned the complete research-to-production pipeline including fine-tuning state-of-the-art models, domain research, service integration, prompt engineering, and inference optimization for speech processing and conversational AI features.}\ifko{음성 처리 및 대화형 AI 기능을 위한 SOTA 모델 파인튜닝, 도메인 리서치, 서비스 통합, 프롬프트 엔지니어링, 추론 최적화를 포함한 전체 연구-프로덕션 파이프라인을 담당했습니다.}}
    }

    \entrybig
    {\link{https://ncsoft.com/}{\textbf{NCSOFT}}}{\link{https://nc-moai.github.io}{Graphics AI Lab}}
    {\ifen{Machine Learning Researcher}\ifko{머신러닝 연구원}}{Feb. 2021 {\textendash} Jun. 2024}
    \innerlist{
        \entry{\ifen{Text-to-Texture Generation System (Dec 2023 - Jun 2024): Built scalable model serving infrastructure and interactive PoC web application (React.js, three.js, FastAPI), evaluating 5+ generative models for natural texture synthesis.}\ifko{Text-to-Texture 생성 시스템 (2023년 12월 - 2024년 6월): 확장 가능한 모델 서빙 인프라와 인터랙티브 PoC 웹 애플리케이션(React.js, three.js, FastAPI)을 구축하고, 자연스러운 텍스처 합성을 위한 5개 이상의 생성 모델을 평가했습니다.}}
        \entry{\ifen{Real-time Digital Human Facial Animation (Feb 2022 - Nov 2023): Built production API server generating persona-aware facial expressions from diverse inputs (audio, text, emotion tags) for conversational scenarios with sub-100ms response time.}\ifko{실시간 디지털 휴먼 얼굴 애니메이션 (2022년 2월 - 2023년 11월): 다양한 입력(오디오, 텍스트, 감정 태그)으로부터 대화 시나리오를 위한 페르소나 인식 얼굴 표정을 생성하는 프로덕션 API 서버를 구축하여 100ms 미만의 응답 시간을 달성했습니다.}}
        \entry{\ifen{Speech-driven 3D Facial Animation Research (Feb 2021 - Jan 2022): Designed PyTorch-based training framework and preprocessing pipelines for motion capture data, reducing experiment setup time by 60\%. Researched sequence-to-sequence architectures for speech-driven animation.}\ifko{음성 기반 3D 얼굴 애니메이션 연구 (2021년 2월 - 2022년 1월): 모션 캡처 데이터를 위한 PyTorch 기반 학습 프레임워크와 전처리 파이프라인을 설계하여 실험 설정 시간을 60\% 단축했습니다. 음성 기반 애니메이션을 위한 시퀀스-투-시퀀스 아키텍처를 연구했습니다.}}
    }

    \entrybig
    {\link{https://clova.ai}{\textbf{Naver Clova}}}{}
    {\ifen{Machine Learning Engineer, Intern}\ifko{머신러닝 엔지니어 인턴}}{Jul. 2020 {\textendash} Aug. 2020}
    \innerlist{
        \entry{\ifen{Contributed to the development of a multi-scale depth estimation network for 3D map reconstruction.}\ifko{3D 지도 재구성을 위한 멀티스케일 깊이 추정 네트워크 개발에 기여했습니다.}}
    }

    \entrybig
    {\link{https://webtoonscorp.com}{\textbf{Naver Webtoon}}}{}
    {\ifen{Machine Learning Researcher, Intern}\ifko{머신러닝 연구원 인턴}}{Sep. 2018 {\textendash} Feb. 2019}
    \innerlist{
        \entry{\ifen{Built an annotation tool for 2D bounding boxes with real-time model inference, query-based dataset management, and flexible tagging capabilities.}\ifko{2D 바운딩 박스를 위한 어노테이션 도구를 구축하고, 실시간 모델 추론, 쿼리 기반 데이터셋 관리, 유연한 태깅 기능을 구현했습니다.}}
        \entry{\ifen{Designed a Visual QA task for an internal workshop, including a multi-modal network and validation dataset.}\ifko{내부 워크숍을 위한 Visual QA 작업을 설계하고, 멀티모달 네트워크와 검증 데이터셋을 개발했습니다.}}
    }

    \entrybig
    {\link{https://swmaestro.org}{\textbf{SW Maestro}}}{}
    {\ifen{Mentee, Awarded Top Team}\ifko{멘티, 최우수팀 수상}}{Jul. 2017 {\textendash} Aug. 2018}
    \innerlist{
        \entry{\ifen{Selected as a top team and showcased at the "100+ Conference" by the Ministry of Science and ICT.}\ifko{최우수팀으로 선정되어 과학기술정보통신부 주최 "100+ 컨퍼런스"에 전시되었습니다.}}
    }
}
